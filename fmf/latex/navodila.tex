% !TeX spellcheck = sl_SI
\documentclass[a4paper,oneside,10pt]{article}

\usepackage[utf8]{inputenc}    % make čšž work on input
\usepackage[T1]{fontenc}       % make čšž work on output
\usepackage[slovene]{babel}    % slovenian language and hyphenation
\usepackage[reqno]{amsmath}    % basic ams math environments and symbols
\usepackage{amssymb,amsthm}    % ams symbols and theorems
\usepackage{graphicx}          % images

\usepackage[
  paper=a4paper,
  top=2.5cm,
  bottom=2.5cm,
  left=2.5cm,
  right=2.5cm
% textheight=24cm,
]{geometry}  % page geomerty

\usepackage{multicol}
\usepackage{minted}
\setminted{fontsize=\small,baselinestretch=1}
\usepackage{mathrsfs}

% clickable references, pdf toc
\usepackage[bookmarks, colorlinks=true, linkcolor=black, anchorcolor=black,
  citecolor=black, filecolor=black, menucolor=black, runcolor=black,
  urlcolor=black, pdfencoding=unicode]{hyperref}

\hypersetup{pdftitle={Praktičen uvod v LaTeX}, pdfauthor={Jure Slak}}

% lastne definicije
\newcommand{\N}{\ensuremath{\mathbb{N}}}
\newcommand{\Z}{\ensuremath{\mathbb{Z}}}
\newcommand{\Q}{\ensuremath{\mathbb{Q}}}
\newcommand{\R}{\ensuremath{\mathbb{R}}}
\renewcommand{\C}{\ensuremath{\mathbb{C}}}
\renewcommand{\emph}[1]{\textbf{#1}}

\newmintinline[ltx]{latex}{}
\newmintinline[ltxs]{latex}{showspaces}

% \pagestyle{empty}              % vse strani prazne
% \setlength{\parindent}{0pt}    % zamik vsakega odstavka
% \setlength{\parskip}{10pt}     % prazen prostor pod odstavkom
\setlength{\overfullrule}{30pt}  % oznaci predlogo vrstico z veliko črnine

\begin{document}

\thispagestyle{empty}
\begin{center}
  \vspace*{\fill}
  \textbf{\huge Praktičen uvod v \LaTeX} \\[2ex]
  {\large Jure Slak}

  \vfill

  \tableofcontents

  \vfill

  \today
\end{center}

\newpage

\section{Osnovno o \LaTeX u}
\LaTeX (izgovorjava ``lateh'') je sistem za pisanje dokumentov, ki ga je razvil Leslie Lamport kot
izboljšavo predhodnega sistema Donalda Knutha, imenovanega \TeX.

\LaTeX{} dokumente se ponavadi piše v \texttt{.tex} datoteke. Najbolj osnovna struktura datoteke
izgleda tako:

\begin{minted}[linenos]{latex}
\documentclass{article}
% preambula
\begin{document}
vsebina
\end{document}
\end{minted}

Na vrhu je definiran tip dokumenta. Običajno se za manjše dokumente uporablja article, obstajajo pa
tudi drugi, pogosto tudi dani s strani inštitucije, za katero dokument pišete.
Temu sledi \emph{preambula}, del dokumenta, ki ni vsebinski, ampak je namenjen različnim
nastavitvam. Nato sledi značka za začetek dokumenta, sledi vsebina in
značka za konec dokumenta. Vse kar sledi tej znački, prevajalnik ignorira.

Tekom prevajanja program \verb|pdflatex| producira različne datoteke \verb|.log|. \verb|.toc|,
\verb|.aux| itd., ki jih lahko po koncu pobrišemo. Tudi če gre med prevajanjem kaj narobe, jih lahko
pobrišemo, da začnemo na novo.

Latex pozna ukaze in okolja:
\begin{itemize}
  \item \emph{ukazi} so oblike \ltx{\ukaz}, \ltx{\ukaz{arg1}}, \ltx{\ukaz{arg1}{arg2}}, itd.\ ali \\
    \ltx{\ukaz[arg]}, \ltx{\ukaz[arg]{arg}}, \ltx{\ukaz[arg]{arg1}{arg2}}, itd.,
    kjer so argumenti \texttt{arg1}, \texttt{arg2} v zavitih oklepajih
    obvezni, argument v oglatih pa neobvezni in jih lahko izpustimo.
  \item \emph{okolja} so oblike \ltx{\begin{okolje} ... \end{okolje}}, \\
    \ltx{\begin{okolje}{arg1} ... \end{okolje}}, itd.\ ali \\
    \ltx{\begin{okolje}[arg] ... \end{okolje}}, \\
    \ltx{\begin{okolje}[arg]{arg1} ... \end{okolje}}, itd., kjer kot prej v zavitih oklepajih
    navajamo obvezne argumente, v oglatih pa neobvezne.
  \item poznamo tudi \emph{stare \TeX{} ukaze}, ki so oblike \ltx{{\ukaz }} in veljajo od ukaza do
    konca trenutnih zavitih oklepajev ali okolja. Kodo znotraj okolji ponavadi zamikamo za 2
    presledka ali pa sploh ne.
\end{itemize}

\emph{Komentarje} pišemo z znakom \%. Prevajalnik ignorira vse od znaka \% do konca vrstice.

Drugi \emph{posebni znaki} so \verb|# $ % ^ & _ { } ~ \|. Če jih želite uporabiti v dokumentu
dobesedno, je potrebno napisati \verb|\# \$ \% \^{} \& \_ \{ \} \~{} \textbackslash|.

V preambuli dokumenta definiramo z ukazom \ltx{\usepackage[options]{package-name}} vse pakete, ki jih bomo uporabljali.
Standardna preambula vsebuje:

\begin{minted}[linenos]{latex}
\documentclass[a4paper,oneside,12pt]{article}

\usepackage[utf8]{inputenc}    % make čšž work on input
\usepackage[T1]{fontenc}       % make čšž work on output
\usepackage[slovene]{babel}    % slovenian language and hyphenation
\usepackage[reqno]{amsmath}    % basic math package
\usepackage{amssymb,amsthm}    % symbols and theorem environments
\usepackage{graphicx}          % images

\usepackage[
  top=2.5cm,
  bottom=2.5cm,
  left=2.5cm,
  right=2.5cm
]{geometry}  % page geomerty

% vstavi svoje pakete tukaj

% clickable references, pdf table of contents
\usepackage[bookmarks, colorlinks=true, linkcolor=black, anchorcolor=black,
  citecolor=black, filecolor=black, menucolor=black, runcolor=black,
  urlcolor=black, pdfencoding=unicode]{hyperref}
\end{minted}

Ukazu \ltx{\documentclass} lahko podamo možnosti za velikost pisave in papirja. Vključimo pakete za
delo z UTF-8 znaki, podporo slovenščini, osnovne matematične pakete in pakete za podporo slik.  S
paketom \verb|geometry| nastavimo odmike na robu. Paket \verb|hyperref| poskrbi za to, da lahko
kliknemo na sklice in citate  v PDF datoteki in da ima ta ob strani strukturo dokumenta.  To lahko
vidite tudi v tem dokumentu.

\section{Osnovna struktura in oblikovanje dokumenta}
Osnovno logično strukturo definiramo z ukazi \ltx{\section}, \ltx{\subsection} in
\ltx{\subsubsection}. Obstajajo tudi verzije ukazov z zvezdico, \ltx{\section*} itd., ki poglavij ne
številčijo.

Besedilo pišemo prosto, pri čemer \LaTeX{} ignorira nove vrstice in zamike presledke v besedilu.
Sam tudi deli besede. Če katere besede ne zna deliti, mu lahko pomagamo z ukazom
\ltx{\hyphenation{ra-ču-nal-nik}} v preambuli.
\emph{Nov odstavek} začnemo tako, da spustimo eno ali več praznih vrstic. Z ukazom \ltx{\\} ali
\ltx{\newline} lahko na silo začnemo novo vrstico, z ukazom \ltx{\newpage} pa novo stran.
Ta ukaza se uporablja sorazmerno redko.

\begin{minted}[linenos]{latex}
  To je odstavkek z vrstico, \\ ki smo jo namenoma zlomili.
  To je še vedno isti odstavek. LaTeX dodatne     presledke
       ali zamike ignorira.

  To je nov odstavek.

  \newpage

  To je nov odstavek na novi strani.
\end{minted}

\emph{Stil strani} lahko nastavljamo z ukazom \ltx{\pagestyle}, ki je najbolj pogosto poklican z
argumentom \texttt{empty} ali \texttt{plain}. Če želimo stil spremeniti samo za eno stran, uporabimo
ukaz \ltx{\thispagestyle}.

Besedilo lahko poravnamo levo, desno ali sredinsko z okolji \verb|flushleft|, \verb|flushright| in
\verb|center|.

\subsection{Velikost in oblika pisave}
\label{sec:font-style}

Za velikost pisave imamo na voljo stare \TeX{} ukaze: \\

\begin{tabular}[h]{ll}
  \ltx{{\Huge         Primer besedila}} & {\Huge         Primer besedila} \\
  \ltx{{\huge         Primer besedila}} & {\huge         Primer besedila} \\
  \ltx{{\LARGE        Primer besedila}} & {\LARGE        Primer besedila} \\
  \ltx{{\Large        Primer besedila}} & {\Large        Primer besedila} \\
  \ltx{{\large        Primer besedila}} & {\large        Primer besedila} \\
  \ltx{{\normalsize   Primer besedila}} & {\normalsize   Primer besedila} \\
  \ltx{{\small        Primer besedila}} & {\small        Primer besedila} \\
  \ltx{{\footnotesize Primer besedila}} & {\footnotesize Primer besedila} \\
  \ltx{{\scriptsize   Primer besedila}} & {\scriptsize   Primer besedila} \\
  \ltx{{\tiny         Primer besedila}} & {\tiny         Primer besedila} \\
\end{tabular} \\[\baselineskip]

\noindent Za obliko pisave imamo na voljo naslednje nove \LaTeX{} ukaze: \\

\begin{tabular}[h]{lll}
  Roman (serif): & \ltx{\textrm{Primer besedila}} & \textrm{Primer besedila} \\
  Sans-serif: & \ltx{\textsf{Primer besedila}} & \textsf{Primer besedila} \\
  Monospaced (typewriter-type): & \ltx{\texttt{Primer besedila}} & \texttt{Primer besedila} \\
  Bold face:& \ltx{\textbf{Primer besedila}} & \textbf{Primer besedila} \\
  Italic: & \ltx{\textit{Primer besedila}} & \textit{Primer besedila} \\
  Small caps: & \ltx{\textsc{Primer besedila}} & \textsc{Primer besedila} \\
  Upright: & \ltx{\textup{Primer besedila}} & \textup{Primer besedila} \\
  Slanted: & \ltx{\textsl{Primer besedila}} & \textsl{Primer besedila} \\
  Medium weight: & \ltx{\textmd{Primer besedila}} & \textmd{Primer besedila} \\
\end{tabular} \\

Ukaze lahko seveda gnezdimo. Za uporabo znotraj matematičnih okolij imamo verzije \ltx{\mathrm},
\dots Za \emph{poudarjene izraze} se ponavadi uporablja ukaz \verb|\emph{ }|, ki privzeto tekst napiše ležeče, dvojno
poudarjen tekst pa zopet pokončno.

\subsection{Prazen prostor}
Praviloma se \LaTeX u pusti, da se sam določa, kam postavi stvari in koliko prostora pusti vmes, pri
čemer imamo na voljo veliko nastavitev.

\begin{description}
  \item[Presledki:] Za piko \LaTeX običajno vstavi daljši presledek kot ponavadi, kar je na koncu
    stavka zaželeno, na koncu okrajšav pa ne.  Zato je potrebno za okrajšavami vstaviti trdi
    presledek \ltxs{\ }. Poleg tega imamo na voljo tudi nedeljivi presledek \verb|~| (zapisan z
    vijugico), na katerem \LaTeX ne bo dovolil preloma vrstice. Ta se lahko uporablja za
    okrajšavami, vedno pa se ga uporablja tudi pred sklici. Obratno, če imamo piko za
    veliko črko, \LaTeX\ ne bo vstavil večjega presledka, vendar mu lahko z ukazom \verb|\@| povemo,
    da ta pika končuje stavek. Primer napačne in pravilne rabe \verb|\@| in \ltxs{\ } vidimo spodaj:

    \scalebox{1.5}{``Rad imam BASIC. Verjetno sem starejši od npr. 50 let.''}

    \scalebox{1.5}{``Rad imam BASIC\@. Verjetno sem starejši od npr.\ 50 let.''}

    Pravilno pisanje okrajšav je torej \ltxs{npr.\ tako} ali \ltx{npr.~tako}. Enako velja za datume.
    Nedeljiv presledek se piše tudi npr.~pri prof.~Golob: \ltx{prof.~Golob}.

  Daljše presledke za piko lahko izklopimo z ukazom \ltx{\frenchspacing}.

  \item[Odstavki:] Nov odstavek privzeto ne preskoči nič prostora, naslednji je le zamaknjen noter.
    To lahko kontroliramo z dolžinama \ltx{\parskip} in \ltx{\parindent}. S kombinacijo ukazov
    \begin{minted}[linenos]{latex}
      \setlength{\parskip}{12pt}
      \setlength{\parindent}{0pt}
    \end{minted}
    lahko \LaTeX{} pripravimo do tega, da pušča pod ostavki 12pt prostora in jih ne zamika, podobno
    kot Word. Če zgolj enega ostavka ne želimo zamakniti, uporabimo \ltx{\noindent}. Pri prelomu
    vrstice lahko kot opcijski argument povemo, koliko prostora naj spusti, npr.\ \ltx{\\[12pt]}

    Preostale dolžine, ki se uporabljajo so še \ltx{\lineskip}, \ltx{\linewidth}, \ltx{\textwidth},
    \ltx{\pagewidth} (in \verb|height| verzije).

  \item[Poljuben prazen prostor:] Za vstavljanje praznega prostora v horizontalni ali vertikalni
    smeti lahko uporabimo ukaza \ltx{\hspace} in \ltx{\vspace}, kot npr.~\ltx{\hspace{10pt}}.
    Prazen prostor se ne vstavi če smo na robu strani. Če želimo vstavljanje praznega prostora tudi
    v tem primeru, uporabimo \ltx{\hspace*}.

  \item[Raztegljiv prazen prostor:] vstavimo lahko tudi prostor, ki se raztegne, kolikor ima
    prostora na strani. To naredimo z \ltx{\hfill} in \ltx{\vfill} ali \ltx{\hspace*{\fill}} in
    \ltx{\vspace*{\fill}}. Zvezdica ima enak pomen kot zgoraj.
    To je uporabno, če želimo nekaj poravnati na sredino strani vertikalno, recimo na naslovnici:

    \begin{minted}[linenos]{latex}
      Univerza v Ljubljani
      \vfill
      Naslov diplome \\
      Avtor diplome
      \vfill
      13.\ 4.\ 2018
    \end{minted}

    ali pa v vrstici desno pri glavi uradnega pisma:

    \begin{minted}[linenos]{latex}
      Fakulteta za matematiko in fiziko \\
      Jadranska 19 \\
      1000 Ljubljana \hfill 20.\ 3.\ 2018 \\
    \end{minted}
\end{description}

\subsection{Seznami}
Za urejene sezname imamo okolje \texttt{enumerate}, za neurejene pa \texttt{itemize}.
Za opisne sezname imamo \texttt{description}. Nov element seznama začnemo z ukazom \ltx{\item},
ki za opcijski argument sprejme besedilo oz.\ znak za začetek te točke seznama.

\begin{minipage}[t]{0.5\linewidth}
\begin{minted}[linenos]{latex}
  \begin{enumerate}
    \item Prvi element.
    \item Drugi element. Seznam ima lahko
      več vrstic.

      In več odstavkov.

    \item Lahko ima tudi gnezdene sezname:
      \begin{itemize}
        \item Gnezned element 1
        \item[$\diamond$] Gnezden element 2
      \end{itemize}
  \end{enumerate}
\end{minted}
\end{minipage}
\begin{minipage}[t]{0.45\linewidth}
\begin{enumerate}
  \item Prvi element.
  \item Drugi element. Seznam ima lahko
    več vrstic.

    In več odstavkov.

  \item Lahko ima tudi gnezdene sezname:
    \begin{itemize}
      \item Gnezned element 1
      \item[$\diamond$] Gnezden element 2
    \end{itemize}
\end{enumerate}
\end{minipage}

\vspace{1ex}

Več opcij glede seznamov ponuja paket \texttt{enumerate}.

\subsection{Ločila in opombe}
Vezaj, \emph{pomišljaj} in dolgi pomišljal naredimo z ukazi \verb|-|, \verb|--| in \verb|---|.

\emph{Narekovaje} naredimo z pari \verb|``text''|, \verb|,,text''|, \verb|>>text<<|, ki izgledajo kot
``text'', ,,text'', >>text<<. Pozor: noben od znakov v kodi ni znak za dvojni narekovaj. Večina
urejevalnikov ima na voljo možnost, da se nastavi ``smart quotes'', ki ob znaku za dvojni narekovaj
vstavijo primerne narekovaje. Podobno naredimo enojne narekovaje \verb|`text'| za `text'.

Tri pike naredimo z ukazom \ltx{\dots}: \dots in ne tako, da napišemo ... ali . . .

Znak \textregistered{} vstavimo z \ltx{\textregistered}.

\emph{Naglase} naredimo z ukazi \verb|\'{o}|, \verb|\`{o}'|, \verb|\"{o}|, \verb|\^{o}|, \verb|\~{o}| za
\'{o}, \`{o}, \"{o}, \^{o}, \~{o}. Lahko jih napišemo tudi direktno v besedilo.
Za \"o je uporabljen znak za dvojni narekovaj. Pišemo jih lahko tudi na kratko kot \verb|\'o|.

\emph{Opombe} pod črto naredimo z ukazom \ltx{\footnote{besedilo}}, kot tukaj.\footnote{Več ukazov za naglase lahko
najdete na \url{https://en.wikibooks.org/wiki/LaTeX/Special_Characters\#Escaped_codes}.}

\emph{Ligature} so skupki več znakov, ki so povezani, za lažje branje. \LaTeX{} jih naredi
avtomatsko, kot npr.\ tukaj:
\scalebox{4}{fiz}
Lahko jih prekinemo, tako da med črke vstavimo prazno škatlico: \ltx{f\mbox{}iz}:
\scalebox{4}{f\mbox{}iz}.

\section{Matematična okolja}
Za pisanje matematičnih formul imamo dva načina: tekstovnega in vrstičnega (med njima lahko tudi
ročno preklapljamo z \ltx{\textstyle} in \ltx{\displaystyle}).

Tekstovni način začnemo in končamo z znakom \verb|$| in vse vmes se izpiše v matematičnem načinu.
\ltx{Naj bo $n$ naravno število} izpiše ``Naj bo $n$ naravno število''. Vsaka spremenljivka mora
biti napisana v matematičnem načinu, sicer izgleda narobe. Primer:

V $i$-ti vrstici algoritem obravnava število $a$.
V i-ti vrstici algoritem obravnava število a.

Vrstični način naredimo z parom \verb|\[ \]|,\footnote{Včasih se je uporabljalo \texttt{\$\$ \$\$}, vendar
ima ta kombinacija lahko težave. Več:
\url{https://tex.stackexchange.com/questions/503/why-is-preferable-to}}
kar postavi enačbo v svojo vrstico. Npr.~\ltx{\[ a+b = c \]} izpiše
\[
 a + b = c
\]

Za številčene enačbe uporabljamo okolje \texttt{equation}, kot \ltx{\begin{equation} a+b =
c \end{equation}}:
\begin{equation}
  a + b = c
  \label{eq:sample}
\end{equation}

V enačbah ne sme biti praznih vrstic, ker bi začele nov odstavek.

\emph{Podpisane} in \emph{nadpisane} znake dobimo kot \ltx{$a^b$} in \ltx{$a_b$}. Za več kot en nadpisan znak je
potrebno uporabiti zavite oklepaje: \ltx{$2^{10}$}, sicer dobimo $2^10$.

\emph{Neenakosti} pišemo z \ltx{$a < b$}, \ltx{$a > b$}, \ltx{$a \leq b$}, \ltx{$a \geq b$}, \ltx{$a
\neq b$}, \ltx{$a \approx b$}: $a<b, a>b, a\leq b, a\geq b, a\neq b, a \approx b$.

\emph{Elementarne funkcije} so vgrajene in jih pišemo kot \\
\ltx{$\sin x, \log(x+y), \sqrt{3}, \sqrt[7]{x+\frac{2 \cdot z}{y}}$}:
$\sin x, \log(x+y), \sqrt{3}, \sqrt[7]{x+\frac{2 \cdot z}{y}}$.

Primerjava: $\sin x$, $sinx$, $\sin(x)$, $sin(x)$.

Če naša izbrana funkcija, npr.~sinc, ni definirana, jo lahko za enkratno uporabo pišemo z ukazom
\ltx{\operatorname{sinc}}, kar zagotovi pravilno računanje presledkov okoli nje,
če pa jo uporabljamo večkrat, se jo splača v preambuli definirati z
\ltx{\DeclareMathOperator{\sinc}{sinc}}, da jo lahko uporabljamo kot $\operatorname{sinc} x$.

\emph{Oklepaje} napišemo kot \ltx{$(x), [x], \{x\}, |x|, \|x\|$} za $(x), [x], \{x\}, |x|, \|x\|$.
Uporaba \ltx{$||x||$} da napačne rezultate, primerjava $\|x\|$ in $||x||$.
Oklepaje lahko tudi skaliramo z \ltx{$\left(\frac{a}{b}\right)$} za
\[
  \left(\frac{a}{b}\right) \quad \text{namesto}  \quad (\frac{a}{b}).
\]

Velike operatorje, kot so
\[
  \sum_{i=1}^n \frac{1}{i} \quad \int_a^b f(x)\,dx \quad \max_{x \in [a, b]} f(x)
  \quad \lim_{n\to\infty} \sqrt[n]{n} \quad \bigcup_{i = 1}^\infty [i, i+1] \times \emptyset
\]
pišemo z
\begin{minted}[linenos]{latex}
  \sum_{i=1}^n \frac{1}{i}
  \int_a^b f(x)\,dx
  \max_{x \in [a, b]} f(x)
  \lim_{n\to\infty} \sqrt[n]{n}
  \bigcup_{i = 1}^\infty [i, i+1] \times \emptyset
\end{minted}

\textbf{Grške} črke pišemo s primernimi ukazi: \\[6pt]
\begin{tabular}[h]{ll|ll}
$\alpha A$                & \ltx{\alpha A}                & $\nu N $                  & \ltx{\nu N }                 \\
$\beta B$                 & \ltx{\beta B}                 & $\xi \Xi $                & \ltx{\xi \Xi }               \\
$\gamma \Gamma$           & \ltx{\gamma \Gamma}           & $o O $                    & \ltx{o O }                   \\
$\delta \Delta$           & \ltx{\delta \Delta}           & $\pi \varpi \Pi $         & \ltx{\pi \varpi \Pi }        \\
$\epsilon \varepsilon E$  & \ltx{\epsilon \varepsilon E}  & $\rho\varrho P $          & \ltx{\rho\varrho P }         \\
$\zeta Z$                 & \ltx{\zeta Z}                 & $\sigma \varsigma \Sigma$ & \ltx{\sigma \varsigma \Sigma}\\
$\eta H$                  & \ltx{\eta H}                  & $\tau T $                 & \ltx{\tau T }                \\
$\theta \vartheta \Theta$ & \ltx{\theta \vartheta \Theta} & $\upsilon \Upsilon $      & \ltx{\upsilon \Upsilon }     \\
$\iota I$                 & \ltx{\iota I}                 & $\phi \varphi \Phi $      & \ltx{\phi \varphi \Phi }     \\
$\kappa K$                & \ltx{\kappa K}                & $\chi X $                 & \ltx{\chi X }                \\
$\lambda \Lambda$         & \ltx{\lambda \Lambda}         & $\psi \Psi $              & \ltx{\psi \Psi }             \\
$\mu M$                   & \ltx{\mu M}                   & $\omega \Omega $          & \ltx{\omega \Omega }         \\
\end{tabular} \\

\textbf{Oznake} nad črkami dobimo z ukazi
\ltx{\bar{z} \vec{\imath} \hat{u} \dot{x} \ddot{y} \tilde{k}}:
$\bar{z}$, $\vec{\imath}$, $\hat{u}$, $\dot{x}$, $\ddot{y}$, $\tilde{k}$.
Ukaza \ltx{\imath} in \ltx{\jmath} naredita $\imath$ in $\jmath$ brez pike, da lahko nanju damo
oznake.

Obstaja tudi ogromno drugih simbolov, ki se jih ponavadi najde iskanjem po spletu oblike ``latex
what is the correct symbol for [insert here]'' ali uporabo celotne tabele simbolov~\cite{a4symb}
ali orodja za prepoznavo simbolov~\cite{detexify}.

\subsection{Presledki}
Presledki v matematičnem načinu delujejo drugače kot v običajnem tekstovnem načinu.  \LaTeX\ ignorira
vse presledke, in se sam odloči, koliko presledka dati okoli črk in opratorjev.

Za lastno vstavljanje presledkov imamo na voljo čedalje večje presledke \ltx{\!}, \ltx{\,},
\ltx{\:}, \ltx{\;}, \ltx{\quad}, \ltx{\qquad}.  Prvi je negativen, \ltx{\,} je majhen presledek,
primeren za npr.~razdaljo med $f(x)$ in $dx$ v integralu $\int f(x)\,dx$, zadnji pa je precej dolg
(dvakrat fontsize). Enak presledek kot med tekstom dobimo z \ltxs{\ }.

\subsection{Poravnava in večji izrazi}
\textbf{Poravnane enačbe} dobimo z okoljem \texttt{align} in \texttt{align*}, pri čemer ukazi z zvezdico
kot ponavadi označujejo verzijo brez številčenja.
Uporaba:
\begin{minted}[linenos]{latex}
\begin{align*}
  \nabla \cdot \vec{B} &= 0 \\
  \nabla \times \vec{E} &= -\frac{1}{c} \frac{\partial B}{\partial t}
\end{align*}
\end{minted}
Rezultat:
\begin{align*}
 \nabla \cdot \vec{B} &= 0 \\
 \nabla \times \vec{E} &= -\frac{1}{c} \frac{\partial B}{\partial t}
\end{align*}

Enačbe so poravnane na znak $\&$, znak \verb|\\| pa označuje novo vrstico.

Opomba: v zadnji vrstici align ne smemo imeti \verb|\\|, sicer bo po enačbah dodaten presledek. Prav
tako pred in po okolju ne smemo imeti praznih vrstic.

\emph{Matrike} pišemo z okolji tipa \texttt{matrix} pri čemer so definirana okolja z začetnimi
črkami \texttt{pbBvV} za različne vrste oklepajev.
\begin{multicols}{2}
  \[
    \begin{bmatrix}
      a & b \\
      c & d
    \end{bmatrix}^{100}
    \begin{Vmatrix}
      1 & \cdots & n \\
      \vdots & \ddots & \vdots \\
      n & \cdots & 1
    \end{Vmatrix}
  \]
  \columnbreak
\begin{minted}[linenos]{latex}
  \[
    \begin{bmatrix}
      a & b \\
      c & d
    \end{bmatrix}^{100}
    \begin{Vmatrix}
      1 & \cdots & n \\
      \vdots & \ddots & \vdots \\
      n & \cdots & 1
    \end{Vmatrix}
  \]
\end{minted}
\end{multicols}

\emph{Funkcije} z več možnosti definiramo s pomočjo okolja \texttt{cases}.
\begin{multicols}{2}
  \begin{align*}
    f&\colon\R \to \R \\
    f(x) &=
    \begin{cases}
      x, & x > 0 \\
      -x & x \leq 0
    \end{cases}
  \end{align*}
  \columnbreak
\begin{minted}[linenos]{latex}
  \begin{align*}
    f&\colon\R \to \R \\
    f(x) &=
    \begin{cases}
      x, & x > 0 \\
      -x & x \leq 0
    \end{cases}
  \end{align*}
\end{minted}
\end{multicols}

\subsection{Oblika besedila}
V matematičnem načiu imamo za pisanje besedila, kot da bi bili v navadnem načinu,
na voljo ukaz \ltx{\text}. Znotraj ukaza \ltx{\text} lahko ponovno uporabljamo \verb|$| in
znotraj zopet \ltx{\text}, kolikorkrat želimo.
Poleg slogov naštetih v razdelku~\ref{sec:font-style},
imamo na voljo še nekaj posebnih slogov, samo za matematični način.

Glavni so \ltx{\mathcal}, ki ponuja pisane (calligraphic) črke, kot so $\mathcal{LPQHUVI}$,
\ltx{\mathbb}, ki ponuja dvojne (blackboard bold) črke, kot $\mathbb{NRCZAH}$ in
\ltx{\mathfrak}, ki ponuja ``fraktur'' verzijo: $\mathfrak{DFOSRL}$ (DFOSRL).
Paket \verb|mathrsfs| ponuja ``script'' verzijo črk, kot so $\mathscr{IMJSLY}$ (IMJSLY).

\section{Slike in tabele}
\subsection{Vstavljanje slik}

Za delo s slikami potrebujemo paket \verb|graphicx|. Slike vključimo z ukazom
\ltx{\includegraphics}, \LaTeX\ pa jih obravnava kot črke enake velikosti, kot je velikost vključene
slike:

To je stavek, kjer smo na sre\includegraphics{slike/surprise_small.png}dini vstavili sliko.

Velikost slike lahko kontroliramo z neobveznimi argumenti, npr.: \\
\ltx{\includegraphics[width=0.5\linewidth]{slike/surprise.png}}. \\
V tem primeru se pogosto uporablja relativne velikosti, kot zgoraj, lahko pa se predpiše tudi
velikost v pikslih ali centimetrih. Višino lahko kontroliramo s parametrom \verb|height|,
na voljo pa imamo tudi veliko drugih možnosti za zrcaljenje, obrezovanje ipd.

Če je možno, naj bodo vključene slike \emph{vektorske}, (format \verb|pdf|, \verb|eps|), saj
se bodo tako skalirale brez izgub. Vključite pa lahko tudi najpogostejše formate bitnih slik, kot
npr.\ \verb|png| in \verb|jpg|.

\subsection{Vstavljanje tabel}
Table naredimo z okoljem \verb|tabular| in jih podobno kot slike \LaTeX\ obravnava kot velike črke.
Primer tabele in njene izvorne kode:

\begin{multicols}{2}
  \begin{minted}{latex}
    \begin{tabular}{l|l|r||c} \hline
      1.\ naloga & 2.\ naloga & 3.\ naloga
        & Skupaj \\ \hline \hline
      1 & 0.75 & 0.25 & 2 \\
      0.5 & 0.25 & 0.25 & 1 \\
      0.5 & 0 & 0.25 & 0.75 \\
      0.25 & 1 & 2 & 3.25 \\
      1 & 2 & 3 & 6 \\ \hline
    \end{tabular}
  \end{minted}
  \columnbreak
  \begin{tabular}{l|l|r||c}
    \hline
    1.\ naloga & 2.\ naloga & 3.\ naloga & Skupaj \\ \hline \hline
    1 & 0.75 & 0.25 & 2 \\
    0.5 & 0.25 & 0.25 & 1 \\
    0.5 & 0 & 0.25 & 0.75 \\
    0.25 & 1 & 2 & 3.25 \\
    1 & 2 & 3 & 6 \\ \hline
  \end{tabular}
\end{multicols}

Kot pri matrikah, znaki \verb|&| ločujejo stolpce, znaki \verb|\\| pa označujejo nove vrstice.
Ukaz \ltx{\hline} naredi horizontalno črto, vertikalne črte in poravnave stolpcev pa so podane kot
prvi argument okolja \texttt{tabular}.

Napisati je možno tudi mnogo bolj zapletene tabele, vendar je verjetno lažje uporabi kakšen
generator \LaTeX\ kode, kot npr.~\cite{table-generator}.

\subsection{Slike in tabele med besedilom}
Ko običajno vstavljamo slike ali tabele, te plavajo med besedilom. Zato imamo na voljo okolji
\verb|figure| in \verb|table|. Običajna koda za vstavljanje slike je:

\begin{minted}[linenos]{latex}
\begin{figure}[h]
  \centering
  \includegraphics[width=0.4\linewidth]{slike/its_something.jpg}
  \caption{Primer vstavljanja slike.}
  \label{fig:its-something}
\end{figure}
\end{minted}

\begin{figure}[h]
  \centering
  \includegraphics[width=0.4\linewidth]{slike/its_something.jpg}
  \caption{Primer vstavljanja slike.}
  \label{fig:its-something}
\end{figure}

To naredi plavajočo sliko, ki jo bo \LaTeX\ poskušal postaviti v dokumentu tja, kjer je definirana
(to pomeni neobvezni parameter \verb|h| -- here; druge možnosti so \verb|t| -- top (na vrhu strani), \verb|b| --
bottom, \verb|p| -- posebna stran). Ukaz \ltx{\centering} vse znotraj okolja \texttt{figure}
postavi na sredino. Nato se vključi slika, pod sliko pa je njen opis. Temu sledi oznaka slike,
uporabljena zgolj interno za sklicevanje. Pozor: ukazov \ltx{\caption} in \ltx{\label} ne smemo
zamenjati med seboj, sicer sklicevanje ne bo delovalo.

Podobno velja za tabele, kjer plavajočo tabelo naredimo z okoljem \texttt{table}.
Pomen vseh ukazov je enak kot pri okolju \texttt{figure}.
Običajen primer uporabe:
\begin{minted}[linenos]{latex}
\begin{table}[t]
  \centering
  \begin{tabular}{l|l|r||c}
    % podatki od zgoraj
  \end{tabular}
  \caption{Primer vstavljanja tabele.}
  \label{tab:primer}
\end{table}
\end{minted}

\begin{table}[t]
  \centering
  \begin{tabular}{l|l|r||c}
    \hline
    1.\ naloga & 2.\ naloga & 3.\ naloga & Skupaj \\ \hline \hline
    1 & 0.75 & 0.25 & 2 \\
    0.5 & 0.25 & 0.25 & 1 \\
    0.5 & 0 & 0.25 & 0.75 \\
    0.25 & 1 & 2 & 3.25 \\
    1 & 2 & 3 & 6 \\ \hline
  \end{tabular}
  \caption{Primer vstavljanja tabele.}
  \label{tab:primer}
\end{table}

\LaTeX\ pogosto ne postavi slik in tabel tja, kamor bi si jih želeli, ker tam ne
vidi dovolj prostora. V praksi se med pisanjem dokumenta s tem ne smemo
obremenjevati, saj se lahko veliko še spremeni, preden končamo. \emph{Na koncu} lahko
popravimo lokacije s spreminjanem velikosti slik ali tabel, s spreminjanem
položaja v kodi in, v skrajnem primeru, z morebitnim kontroliranjem prostora z
\ltx{\vspace}. Če je slik res veliko, lahko tudi vstavimo prelom strani, ki pred
začetkom nove strani vstavi še vse do zdaj sprocesirane slike in tabele. To
naredimo z ukazom \ltx{\clearpage} (z razliko od \ltx{\newpage}, ki samo gre na
novo stran in ne pomaga pri tem, da bi vstavili kakšno sliko več).

Paket \texttt{caption} omogoča več nadzora nad izgledom in pozicijo slik/tabel
in pripadajočih napisov.

\section{Sklicevanje in kazala}
Avtomatko \emph{sklicevanje} v \LaTeX u predstavlja zelo močno orodje. Skličemo se
lahko na vsak števec -- števcev pa je zelo veliko. \LaTeX ima števce za strani,
razdelke, podrazdelke, enačbe, slike, tabele, oštevilčene sezname, \dots

Oznako števca naredimo z ukazom \ltx{\label}, na oznako pa se skličemo z ukazom
\ltx{\ref}. Ukaz \ltx{\label} si zapomni vrednost zadnjega števca, ki je bil
spremenjen, za to običajno \ltx{\label} napišemo tik za ukazom, ki poveča kakšen
števec. Za slike in tabele tako \ltx{\label} vstavimo tik za \ltx{\cation}, za
razdelke tik za \ltx{\section}, pri seznamih tik za \ltx{\item}. Dodatno velja
nenapisan dogovor, da se oznake poimenuje s predpono, ki pove, na kaj se oznaka
nanaša: oznake za slike, se začnejo s \ltx{\label{fig:oznaka}}, oznake na tabele
s \texttt{tab:}, oznake za razdelke s \texttt{sec:}, oznake za enačbe z
\texttt{eq:}, oznake za elemente seznamov z \texttt{itm:}, oznake za algoritme z
\texttt{alg:} in oznake za izseke izvorne kode z \texttt{lst:}. Uporaba teh oblik
ni obvezna, je pa zelo uporabna pri velikih dokumentih, da mi (ali pa naš
urejevalnik) lažje iščemo med obstoječimi oznakami.

Na številčene elemente se vedno sklicujemo z \ltx{\ref} ali podobnimi ukazi,
nikoli s konkretnimi številkami. Pred ukazom \ltx{\ref} moramo tudi običajno
vstaviti nedeljiv presledek, da se sklic ne prelomi v novo vrstico. Pravilno
sklicevanje na razdelek o obliki pisave, bi bilo tako: \\
\ltx{O obliki pisave smo govorili v razdelku~\ref{sec:font-style}, kjer [se
nadaljuje]}, \\ kar v dokumentu izgleda kot ``O obliki pisave smo govorili v
razdelku~\ref{sec:font-style}, kjer [se nadaljuje]''. Pri tem smo tik po ukazu
\ltx{\subsection{Velikost in oblika pisave}} napisali
\ltx{\label{sec:font-style}}.

Sklicevanje na \emph{stran}, kjer se nahaja oznaka, se naredi z ukazom \ltx{\pageref}.
Sklicali bi se kot: \\
\ltx{Slika~\ref{fig:its-something} na strani~\pageref{fig:its-something} je
lepa.},
kar v dokumentu izgleda kot: ``Slika~\ref{fig:its-something} na
strani~\pageref{fig:its-something} je lepa.''

Pri sklicevanju na enačbe, običajno uporabljamo okrogle oklepaje, kot npr.~pri
enačbi~\eqref{eq:sample}. Sklic z avtomatskimi oklepaji okrog reference dobimo z
ukazom \ltx{\eqref}. Sklic malo prej smo dobili z ukazom
\ltx{\eqref{eq:sample}}, pri čemer smo pred takoj po \ltx{begin{equation}}
  vstavili \ltx{\label{eq:sample}}.

\subsection{Kazala}
Kazalo vsebine vstavimo z ukazom \ltx{\tableofcontents}. Uporabimo lahko še ukaza
\ltx{\listoffigures} in \ltx{\listoftables}.

Vnosi v kazalu slik in tabel so opisi pod slikami in tabelami. Če so ti
predolgi, lahko uporabimo pri ukazu \ltx{caption}, neobvezen argument:
\ltx{\caption[Kratek opis]{Daljši opis.}}, pri čemer se kratek opis pojavi v
kazalu, dolg opis pa pod sliko/tabelo.

Vsi izrazi so že prevedeni v slovenščino s paketom \texttt{babel}. Če nam kakšen
prevod ni všeč, ga lahko spremenimo npr.~iz privzetega ``Slike'' na ``Kazalo slik''
z \\ \ltx{\addto\captionsslovene{\renewcommand\listfigurename{Kazalo slik}}}.

Več možnosti za oblikovanje kazala ponuja paket \texttt{tocloft}.

\section{Literatura}

Za literaturo se v večjih dokumentih uporablja sistem
$\mathrm{B{\scriptstyle{IB}} \! T\!_{\displaystyle E} \! X}$, ki omogoča
avtomatsko generiranje seznama referenc s predpisanim stilom (obstaja
inštitucije ponudijo svojega, tudi FMF). Za kratko navajanje na roko
pa imamo na voljo okolje \texttt{thebibliography}.

Okolje se uporablja takole:

\begin{minted}[linenos]{latex}
\begin{thebibliography}{99}
  \bibitem{oznaka} Poljubno besedilo.
  \bibitem{latex} Lamport, Leslie. \textit{LaTeX: A Document
    Preparation System, 2/e}. Pearson Education India, 1994. URL:
    \url{http://cds.cern.ch/record/270275/files/9780201529838_TOC.pdf}
\end{thebibliography}
\end{minted}

Primer uporabe lahko vidite na koncu tega dokumenta.
Na literaturo se sklicujemo z ukazom \ltx{\cite{oznaka}}.
Kot pri referencah, je tudi tukaj prej pravilno vstaviti nedeljiv presledek
$\sim$. Sklicujemo se lahko na več virov naenkrat, kot
npr.\ \ltx{\cite{a4symb,detexify}}, kar izgleda kot~\cite{a4symb,detexify}.
Citiramo lahko tudi bolj natančno, tako da v oglatih oklepajih podamo dodatno
vsebino: \ltx{\cite[str.\ 12]{a4symb}} izpiše~\cite[str.\ 12]{a4symb}.

\section{Razno (algoritmi, izvorna koda, več stolpcev)}
Za pisanje algoritmov priporočam paketa \texttt{algorithm} in
\texttt{algorithmicx} (potrebujete oba). Za vključevanje in barvanje izvorne
kode (kot v tem dokumentu) je uporabljen paket \texttt{minted}.
Za več stolpcev se lahko uporabi paket \texttt{multicols}.

\begin{thebibliography}{99}
  \bibitem{latex} Lamport, Leslie. \textit{LaTeX: A Document
    Preparation System, 2/e}. Pearson Education India, 1994. URL:
    \url{http://cds.cern.ch/record/270275/files/9780201529838_TOC.pdf}
  \bibitem{a4symb} Scott Pakin. \textit{The Comprehensive \LaTeX\ Symbol List}.
    URL: \url{http://tug.ctan.org/info/symbols/comprehensive/symbols-a4.pdf}
  \bibitem{latexeng} Tobias Oetiker. \textit{The Not So Short
    Introduction to \LaTeX\ $2\varepsilon$}. URL: \url{https://tobi.oetiker.ch/lshort/lshort.pdf}
  \bibitem{latexslo} Bor Plestenjak (prevod in priredba).  \textit{Ne najkrajši
    uvod v\LaTeX\ $2\varepsilon$}.
    URL: \url{http://www-lp.fmf.uni-lj.si/plestenjak/vaje/latex/lshort.pdf}
  \bibitem{detexify} Daniel Kirsch. \textit{Detexify \LaTeX\ handwritten symbol recognition.} URL: \url{http://detexify.kirelabs.org/classify.html}
  \bibitem{table-generator} \textit{Tables Generator}. URL: \url{https://www.tablesgenerator.com/}
\end{thebibliography}

\end{document}
% vim: syntax=tex
% vim: spell spelllang=sl
% vim: foldlevel=99
% Latex template: Jure Slak, jure.slak@fmf.uni-lj.si
